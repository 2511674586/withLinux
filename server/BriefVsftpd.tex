\documentclass[11pt]{article}
%Gummi|065|=)
\title{\textbf{A brief vsftpd guide}}
\author{CDLuminate}
\date{2014/03/23}
\begin{document}

\maketitle

\section{Before you start}

Before you start \emph{vsftpd} server, you should first install it properly.For example on Debian/Ubuntu, just type (\# apt-get install vsftpd)then it will be ready to work.


\section{Check its config file}
Before launch it right away, you should check its configuration file first, to make it more usable.

The [config file]\footnote{file:///etc/vsftpd.conf} contains varies of options, but there are some that may be modified frequently.
 
They are listed below.\\
listen=YES	// if vsftpd would listen via ipv4\\
listen\_ipv6=NO  // as above, via ipv6\\
anonymous\_enable=YES  // if anonymous user can login\\
local\_enable=YES  // if local user can login\\
write\_enable=NO // this is global write setting, if set to NO, then no one can write even if he is allowed to write.\\

\section{Prepare the ftp service directory}
The [default service dir]\footnote{dir:///srv/ftp} of vsftpd contains nothing. So there is 2 ways to serve:

1.Copy files to /srv/ftp directly.

2.mount it here.(see Appendix B).

\section{Start server}
(\# service vsftpd restart). then it would work except that something had gone wrong.

\section{Access server}
It is easy on an AtoB network.\\
you can login via your file browser or internet browser.\\
just navigate to (ftp://192.168.0.252)(in my example).\\

Have fun play with ftp.

\section{Appendix A. An example - Set up a simple AtoB network}

1. connect machines via a network cable.\\
2. stop the "network-manager" service on both side.\footnote{\# service network-manager stop}\\
(choose one of step3)\\
3.1 ComputerA\\
(\# ifconfig eth0 inet 192.168.0.252 netmask 255.255.255.0)\\
(\# ifconfig eth0 up)\\
ComputerB\\
Do the same as ComputerA, changing its ip\_address.\\
3.2 ComputerA\\
(\# ip addr add 192.168.0.252/24 brd + dev eth0)\\
(\# ip link set dev eth0 up)\\
Do the same on computerB.\\
4. Run a ping test.

\section{Appendix B. An mount example}
1. Assume that you want to share this folder (/some/dir)\\
2.(\# mount --bind /some/dir /srv/ftp)\\
3.then you can start you vsftpd.

\end{document}
