\documentclass[a4paper,12pt]{article}

\usepackage{CJKutf8}
\usepackage{CJKulem}

\begin{document}
\begin{CJK}{UTF8}{gbsn}

\section*{HP服务器操作系统安装记录}
	\today, lumin,内容补全的版本。

\section{概要}
	型号
	\begin{verbatim}
	HP ProLiant DL388e Gen8
	\end{verbatim}
	存在驱动问题的阵列卡主控型号
	\begin{verbatim}
	HP Dynamic Smart Array B320i Controllers
	\end{verbatim}
	目标操作系统
	\begin{verbatim}
	CentOS 7
	\end{verbatim}
	原操作系统
	\begin{verbatim}
1. Windows Server 2008
2. Failure Install Instance
	\end{verbatim}

\section{背景}
	根据之前的安装经历,Fedora没有阵列卡驱动因此不能安装。
	本次安装使用的CentOS7官方源中同样没有对应驱动,
	于是从HP官网弄了一个针对RHEL7的驱动(RHEL7只有AMD64架构)。
	\newline
	安装前没有改动过BIOS设置,没有更改过RAID设置。
	原RAID设置为RAID5。

\section{过程概述}
	\subsection{引导安装介质}
	首先用dd 一个 centos 7镜像到U盘中。
	\begin{verbatim}
# dd if=/iso/CentOS-7.0-1406-x86_64-DVD.iso of=/dev/sdc
	\end{verbatim}
	插入U盘后,在服务器的Boot menu中选择从USB设备启动。
	\newline
	引导起来后,内核提示
	\begin{verbatim}
[Firmware Bug] ??? BIOS corrupted hw-PMU
	\end{verbatim}
	具体可用 dmesg 观察,但暂不明其影响。
	将驱动(以及rpm包解压出来的文件,备用)放到另一个U盘中,在live环境下mount到某个目录,然后想办法把该内核模块弄到 /lib/modules 下边,让它能够被加载。内核模块文件为\fbox{hpvsa.ko}
	\begin{verbatim}
# mkdir -p /tmp/hp/hpvsa
# cp hpvsa.ko /tmp/hp/hpvsa
# mount --bind /tmp/hp /lib/modules/3.???/extra
# depmod -a
# modprobe hpvsa
	\end{verbatim}
	至此模块hpvsa已正常加载,而且已经可以通过阵列卡来操作硬盘了。

	\subsection{安装系统}
	由于调整RAID设置需要使用服务器本身自带的程序调整RAID设置,所以这次直接将系统盖在原本Windows使用的分区上,大小为100GB,正常情况下足够使用。
	\newline
	驱动加载好之后图形安装程序似乎仍然不能正常识别硬盘。
	于是到tty2手动执行红帽系的安装程序:
	\begin{verbatim}
# anaconda
	\end{verbatim}
	文字模式下的anaconda可以正常安装,后续过程略。需要注意的是,CentOS7以及RHEL7默认文件系统是XFS。

	\subsection{后期处理}
	如果不安装该模块/软件包,那么系统可以正常启动bootloader,可以正常加载linux内核以及initramfs,但是因为initramfs中没有hpvsa.ko,而不能正常挂载设备,因此导致boot过程卡在initramfs中。
	\newline
	安装过后,手动mount一些特殊目录(原因可见错误提示,或者Gentoo的安装手册),然后拷贝HP官方驱动到临时目录,最后chroot到目标系统中安装驱动。下边的目录仅供参考,具体路径需要取决实际情况。(lsblk)这其实就是在Live环境里给target操作系统安装软件包而已。
	\begin{verbatim}
LIVE:
# mount /dev/sdc3 /target                # sdc3 is /
# mount /dev/sdc1 /target/boot           # sdc1 is /boot
# mount --bind /sys /target/sys
# mount --bind /proc /target/proc
# mount --bind /dev /target/dev
# cp kmod-hpvsa-???.rpm /target/tmp/
# chroot /target /bin/bash
CHROOT:
# cd /tmp
# rpm -i kmod-hpvsa-???.rpm
	\end{verbatim}
	安装完之后控制脚本会自动depmod并更新initramfs,它会将该驱动自动添加到initramfs中,于是理论上便能正常启动。
	如果需要移除软件包,执行 rpm -e <PKG> 即可。

	\subsection{冒烟测试}
	重启,正常启动。ssh已自带。另外,使用
	\begin{verbatim}
# dhclient IFACE
	\end{verbatim}
	可以用dhcp接入网络,其中的 IFACE 可以用命令" ip addr "查看,或者" ifconfig ".

\section{其他}
\begin{enumerate}
	\item hpvsa驱动这个软件包,其内容不太规范,只有两个文件,一个hpvsa.ko,另一个是depmod.d的配置文件。于是只能通过modinfo来查看模块本身的信息。
	\item 这个安装过程并不奇葩,实质上就是所谓的“在安装过程中加载缺失的固件”之类的问题,只是有点Gentoo风格罢了。
	\item dmesg输出的错误信息值得留意,正如上文记载。不过当时没有搜索到有关hw-PMU的信息。
	\item 不知道在 PXE+kickstart 条件下是否可以加载额外的驱动。
\end{enumerate}

\end{CJK}
\end{document}

